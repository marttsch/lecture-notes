\documentclass[mfit.tex]{subfiles}
\begin{document}

\begin{proof}
  $(v) \Rightarrow (iii),(iv)$\\
  $U_{x,x} \overset{(c) z=1}{=} \sum_y p_{x,y} \underbrace{F_{y,x}}_{1} = 1$ for all $x$\\
  $(iv) \Rightarrow (v)$\\
  fix $y$. Then $h(x) = F_{x,y}$. By (d), $h(x) = \sum_w p_{x,w} h(w)$ $x \neq y$.
  By (c), $F_{y,y} = 1 = U_{y,y} = \sum_w p_{y,w} F_{w,y}$
  Then $h(y) = \sum_w p_{y,w} h(w)$
  $P h = h$ for $0 < h \leq 1 = h(y)$
  Thus, $h$ is constant and $h \equiv 1$
  
  $\expect_x(t^x) = U_{x,x}^\prime(1-)$.
  Then there exists an $x$ such that $U_{x,x}^\prime (1-) < \infty \iff \forall x$.
  $M_{x,y} \geq \frac{G_{x,x}(z)}{G_{y,y}(z)} = \frac{1 - U_{y,y}(z)}{1- U_{x,x}(z)}$
  
  $\mathcal{X}$ finite $P$ irreducible, then it is positive recurrent $\expect_x(t^x) < \infty$ for all $x$
  
  \begin{align*}
    \sum_y G_{x,y}(z) = \frac{1}{1-z}
  \end{align*}
  Since $z \to 1-$, there exists a $y$ such that $G_{x,y} = \infty$ is recurrent
  \begin{align*}
    1 = \sum_y F_{x,y}(z) \frac{1-z}{1-U_{y,y}(z)}
  \end{align*}
  since $z \to 1-$,
  \begin{align*}
    1 = \sum_y 1 \frac{1}{\expect_y(t^y)}
  \end{align*}
  There exists a $y$ such that $\expect_y(t^y) < \infty$, thus for all $y$.
  
  \begin{align*}
    \nu_y = \frac{1}{\expect_y(t^y)} > 0 
  \end{align*}
  probability distribution on $\mathcal{X}$.
\end{proof}

\begin{theorem}[Ergodensatz für endliche irreduzible MK]
  Let $f: \mathcal{X} \to \R$. Then for every starting distribution
  \begin{align*}
    \frac{1}{n} \sum_{k=0}^{n-1} f(X_k(\omega)) \overset{a.s.}{\to} \int_\mathcal{X} f d\nu = \sum_x p(x) \nu_x
  \end{align*}
  In particular, $\nu$ is the unique stationary probability measure.
\end{theorem}

\begin{proof}
  Let $\mu$ be the starting distribution. Then define a sequence of stopping times as follows: Choose a fix $y \in \mathcal{X}$.
  \begin{align*}
    \tau_0 &= 0 \\
    \tau_n &= \inf \{k > \tau_{n-1}: X_k = y \}
  \end{align*}
  Thus (Note that start distribution is $\mu$),
  \begin{align*}
    \tau_1 &= t^y < \infty \text{ a.s.}
    \tau_n &< \infty \text{ a.s.}
  \end{align*}
  Look at the increments: $\sigma_n = \tau_n - \tau_{n-1}$ for $n \geq 1$.
\end{proof}

\begin{lemma}
  Let $(\sigma_n)_{n \in \N}$ be independent. 
  And for $n \geq 2$ let $(\sigma_n)_{n \in \N}$ be distributed like $t^y$ at the start in $y$.
\end{lemma}

\begin{proof}
  By \enquote{Händewackeln}
\end{proof}

Continue proof of theorem:
\begin{proof}
  $\tau_n = \sigma_1 + \sum_{k=2}^n \sigma_k$
  We know $\expect(\sigma_k) = \expect_y(t^y) < \infty$.
  Then by LLN, 
  \begin{align*}
    \frac{1}{n} \tau_n = \frac{\sigma_1}{n} + \frac{n-1}{n} \frac{1}{n-1} \sum_{k=2}^n \sigma_k
  \end{align*}
  Then this tends to $\expect_y(t^y)$ almost surely. (for all $\omega \in \Omega_0 \in \mathcal{A}$: $\prob_\mu(\Omega_0) = 1$)
  
  Therefore, $\tau_n$ tends to $\infty$ for $n \in \N$.
  For any $n \in \N$ there exists a $k_n = k_n(\omega)$ such that $\underbrace{\tau_{k_n}(\omega)}_{\tau_{k_n(\omega)}(\omega)} \leq n < \tau_{k_n + 1} (\omega)$.
  If $n \to \infty$, then $k_n(\omega) \to \infty$.
  \begin{align*}
    \frac{\tau_{k_n}}{k_n} \leq \frac{n}{k_n} \leq \frac{\tau_{k_n+1}}{k+1} \frac{k_n + 1}{k_n}
  \end{align*}
  For $n \to \infty$ $\frac{\tau_{k_n}}{k_n} \to \expect_y(t^y)$ and $\frac{\tau_{k_n+1}}{k_n+1} \to \expect_y(t^y)$ and $\frac{k_n+1}{k_n} \to 1$
  Thus, $\frac{k_n}{n} \to \frac{1}{\expect_y(t^y)}$.
  
  \begin{align*}
    V_k^y = \mathbb{1}_{[X_k = y]} \\
    \sum_{k=1}^n V_k^y = k_n \\
    \implies \frac{1}{n} \sum_{k=1}^n V_k^y \to \frac{1}{\expect_y(t^y)} \text{ a.s.}
  \end{align*}
  Whether we consider $\sum_{k=1}^n V_k^y$ or $\sum_{k=0}^n V_k^y$ does not influence the Grenzwert
  
  Let $f: \mathcal{X} \to \R$. Then
  \begin{align*}
    f = \sum_{y \in \mathcal{X}} f(y) \mathbb{1}_y \\
    f(X_k) = \sum_y f(y) V_k^y
  \end{align*}
  \begin{align*}
    \frac{1}{n} \sum_{k=0}^{n-1} f(X_k) = \frac{1}{n} \sum_{k=0}^{n-1} \sum_{y \in \mathcal{X}} f(y) V_k^y \\
  \end{align*}
  Since we have finite sums we can exchange them
  \begin{align*}
    = \sum_{y \in \mathcal{X}} f(y) ( \frac{1}{n} \sum_{k=0}^{n-1} V_k^y) \to \sum_{y \in \mathcal{X}} f(y) \frac{1}{\expect_y(t^y)} \text{ a.s.}
  \end{align*}
  \begin{align*}
    \expect_\mu (\frac{1}{n} \sum_{k=0}^{n-1} V_k^y) \overset{n \to \infty}{ \to} \expect_\mu(\nu_y) = \nu_y
  \end{align*}
  is true by Lebesgue!
  
  Let $\mu = \delta_x$.
  \begin{align*}
    \frac{1}{n} \sum_{k=0}^{n-1} p_{x,y}^{(k)} \to \nu_y
  \end{align*}
  for all $y$
  \begin{align*}
    \frac{1}{n} \sum_{k=0}^{n-1} \delta_y P^k \to \nu
  \end{align*}
  This results in
  \begin{align*}
    \frac{1}{n} \sum_{k=0}^{n-1} \mu P^k \to \nu
  \end{align*}
  for all $\mu$ starting distribution
  
  Let $\mu P = \mu$, then
  \begin{align*}
    \frac{1}{n} \sum_{k=0}^{n-1} \mu P^k = \mu \to \nu
  \end{align*}
  $\mu = \nu$.
\end{proof}

\begin{rem}
  Wenn \enquote{Ergdoensatz für endl. pos rekurrent irred MK}, dann bei GW und summen vertauschungen aufpassen
  $f \in L^1(\mathcal{X},\nu)$ und
  \begin{align*}
    \sum_x \abs{f(x)} \nu_x < \infty
  \end{align*}
  Schwieriger Fall: $\sum_{y \in \mathcal{X}} f(y) ( \frac{1}{n} \sum_{k=0}^{n-1} V_k^y)$
  wenn $f$ keinen endlichen Träger hat (allg Ergodensatz)
\end{rem}

\end{document}
