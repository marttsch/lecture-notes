\documentclass[mfit.tex]{subfiles}
\begin{document}

\section{Mathematical proofs}

\subsection{Birkhoff's Ergodic Theorem and the LLN}

\begin{defi*}
  $T: (\Omega,\mathcal{A},\prob) \to (\Omega,\mathcal{A},\prob)$ \emph{measure preserving} (maßtreu) if
  \[ \prob(\inv{T} A) = \prob(A) \;\; \forall A \in \mathcal{A} \text{.} \]
  This statement is equivalent to
  \[ \expect(f \circ T) = \expect(f) \;\; f \in L^1(\Omega,\mathcal{A},\prob) \]
  $T$ is called \emph{ergodic} if $\inv{T}A = A \implies \prob(A) \in \{0,1\}$ for $A \in \mathcal{A}$.
  \[ \mathcal{I} = \{ A \in \mathcal{A}: \inv{T}A = A \text{ invariant } \sigma \text{-algebra} \} \]
\end{defi*}

\begin{rem}
  For this definition there is no significant difference if we replace $\inv{T}A=A$ by $\inv{T}A \overset{\text{a.s.}}{=}A$.
\end{rem}

\begin{theorem}[Birkhoff's Ergodic Theorem]
  $T$ measure preserving, $f \in L^1$, $S_n(f) = \sum_{k=0}^{n-1} f \circ T^k$.
  Then there exists $M(f) \in L^1(\Omega,\mathcal{I},\prob)$ such that $\frac{1}{n} S_n(f) \to M(f)$ almost surely and 
  \[ \int_A f d\prob = \int_A M(f) d\prob \;\; \forall A \in \mathcal{I} \]
  that is $M(f) = \expect(f|_\mathcal{I})$.
  If $T$ is ergodic, then $\mathcal{I}$ is trivial and $\expect(f|_\mathcal{I}) = \expect(f)$.
\end{theorem}

\begin{theorem}[von Neumann's Ergodic Theorem]
  $T$ measure preserving, $f \in L^1$, $S_n(f) = \frac{1}{n} \sum_{k=0}^{n-1} f \circ T^k$.
  Then there exists $M(f) \in L^1(\Omega,\mathcal{I},\prob)$ such that $S_n(f) \to M(f)$ almost surely and in $L^1$
  \[ \int_A f d\prob = \int_A M(f) d\prob \;\; \forall A \in \mathcal{I} \]
  that is $M(f) = \expect(f|_\mathcal{I})$.
  If $T$ is ergodic, then $\mathcal{I}$ is trivial and $\expect(f|_\mathcal{I}) = \expect(f)$.
\end{theorem}

How to deduce the Law of Large Numbers from Birkhoff's ergodic theorem?

$(X_n)$ iid, real RVs, $\expect(\abs{X_k}) < \infty$
\[ S_n = X_1 + \dots + X_n \overset{?}{\implies} \frac{1}{n} S_n \overset{\text{a.s.}}{\to} \expect(X_1) \text{.} \]

$P_X$: distribution of $X_k$.
$(\R,\mathcal{B},P_X)$, $X(\omega) = \omega_1$, $\omega_1 \in \R$.

\[ (\Omega,\mathcal{A},\prob) = \otimes_{n=1}^\infty (\R,\mathcal{B},P_X) \]
\[ \Omega = \R^\N = \{ \omega = (\omega_1,\omega_2,\dots): \omega_k \in \R \} \]
$\mathcal{A}$: $\sigma$-algebra generated by all sets
\[ C(I_1,\dots, I_n) = \{ \omega: \omega_k \in I_k \forall k \leq n \} \]
where $I_1,\dots, I_n$ are intervals
Take a $\sigma$-algebra created by all sets $I_1\times I_2 \times \dots \times I_n \times \R \times \R \times \dots$.

\[ \prob(C(I_1,\dots,I_n)) = P_X(I_1) \cdots P_X(I_n) \]

Recall: if $(\tilde{\Omega}, \tilde{\mathcal{A}}, \tilde{\prob})$ is any probability space and $\tilde{X_n}$ is any sequence of iid RVs with distribution $P_X$.

\todo{add pic}

$\tau(\tilde{\omega}) = (\tilde{X_n}(\tilde{\omega}))_{n\in \N} = (\tilde{X_1}(\tilde{\omega}), \tilde{X_2}(\tilde{\omega}),\dots)$

$T(\omega_1,\omega_2,\dots) = (\omega_2,\omega_3,\dots)$ is measure preserving.
$X_n(\omega_1,\omega_2,\dots) = \omega_n$ and $f = X_1 \in L^1$.
Then we have $f \circ T^{k-1} = X_k$.
Thus,
\[ S_n(f) = \sum_{k=1}^n X_k \]
and so by Birkhoff's Ergodic Theorem,
\[ \frac{1}{n} \overset{\text{a.s}}{\to} \expect(X_1|_\mathcal{I} \]

Now we just need to justify why $T$ is ergodic.
$0$-$1$-law of Kolmogorov:

\begin{theorem}
  $(\Omega,\mathcal{A},\prob)$, $(X_n)$ independent RVs
  Then 
  \[ \underbrace{\tau(X_1,X_2,\dots)}_{\bigcap_{n=1}^\infty \sigma(X_n,X_{n+1},\dots)} \]
  is trivial, i.e. $A \in \tau \implies \prob(A) \in \{0,1\}$. 
  $\tau$ is called the tail $\sigma$-algebra.
\end{theorem}

The argument is 
\begin{align*}
  A \in \tau &\implies A \in \sigma(X_{n+1},X_{n+2},\dots) \;\; \forall n \\
  &\implies A \text{ independent of } \sigma(X_1,\dots,X_n) \\
  &\implies A \text{ independent of } \bigcup_{n=1}^\infty \sigma(X_1,\dots,X_n) \\
  &\implies A \text{ independent of } \bigvee_{n=1}^\infty \sigma(X_1,\dots,X_n) = \sigma(X_1,X_2,\dots) \ni A
\end{align*}

Since $\inv{T}A = A$, $\underbrace{T^{-n}A}_{\in \sigma(X_{n+1},X{n+2},\dots)} = A$.
Thus, $\mathcal{A} \in \tau$.

\subsection{Theorem: Axiom definition of entropy}

\begin{proof}
  \begin{align*}
    H(1,0) \overset{\text{axiom }4}{=} H(1) + 1 H(1,0) \implies H(1) = 0
  \end{align*}
  \begin{align*}
    H(p_1,\dots,p_N,0) &\overset{\text{axiom }4}{=} H(p_1,\dots,p_N) + p_N H(1,0) \\
    H(p_1, 1-p_1, 0) &= H(p_1,1-p_1) + (1-p_1)H(1,0) \\
    H(1-p_1, p_1,0) &= H(1- p_1,p_) + p_1 H(1,0)
  \end{align*}
  Since by axiom 1, $H(p_1, 1-p_1,0) = H(1-p_1,p_1,0)$ and since $H(p_1,1-p_1) = H(1-p_1,p_1)$, it follows that
  \[ (1-p_1) H(1,0) = p_1 H(1,0) \;\; \forall p_1 \]
  Thus, $H(1,0) = 0$.
  
  \begin{claim}
    For all $N$ and for all $m \geq 2$:
    \[ H(p_1,\dots,p_N,p_{N+1},\dots,p_{N+m}) = H(p_1,\dots,p_N,q) + q H\left( \frac{p_{N+1}}{q}, \dots, \frac{p_{N+m}}{q} \right) \]
    where $q = \sum_{k=N+1}^{N+m} p_k$.
  \end{claim}

  \begin{proof}
    Induction: $m=2$ holds by axiom $4$.

    $2 \leq m-1 \to m$:\\
    \begin{align*}
      H&(p_1,\dots,p_N,p_{N+1},\dots,p_{N+m-1},p_{N+m}) \\
      &\overset{\text{axiom }4}{=} H(p_1,\dots,p_N,\dots,p_{N+m-2},p_{N+m-1},p_{N+m}) 
      + \underbrace{(p_{N+m-1}+p_{N+m})}_{q^\prime} H\left( \underbrace{\frac{p_{N+m-1}}{q^\prime}}_{p^\prime}, \underbrace{\frac{p_{N+m}}{q^\prime}}_{1-p^\prime} \right) \\
      &\overset{\text{ind. hyp.}}{=} H(p_1,\dots,p_N,q) + q H \left( \frac{p_{N+1}}{q},\dots,\frac{p_{N+m-2}}{2}, \frac{q^\prime}{q} \right) + q^\prime H(p^\prime,1-p^\prime) \\
      &\overset{\text{axiom }4}{=} H(p_1,\dots,p_N,q) + q \left[ H \left( \frac{p_{N+1}}{q},\dots,\frac{p_{N+m-2}}{q},\frac{p_{N+m-1}}{q},\frac{p_{N+m}}{q} \right) - \frac{q^\prime}{q} H(p^\prime,1-p^\prime) \right] + q^\prime H(p^\prime, 1- p^\prime \\
      &= H(p_1,\dots,p_N,q) + q H \left( \frac{p_{N+1}}{q},\dots,\frac{p_{N+m-2}}{q},\frac{p_{N+m-1}}{q},\frac{p_{N+m}}{q} \right)
    \end{align*}
    Since $- q \cdot \frac{q^\prime}{q} H(p^\prime,1-p^\prime) + q^\prime H(p^\prime, 1- p^\prime = 0$.
  \end{proof}
  
  \begin{claim}
    \[ pq^\prime H(p^\prime, 1- p^\prime{(1)} = (p_1^{(1)},\dots,p_{N_1}^{(1)}),\dots,p^{(n)} = (p_1^{(n)},\dots,p_{N_n}^{(n)}) \in \mathcal{P}\]
    \[ q = (q_1,\dots,q_n) \in \mathcal{P}\]
    \[ (q_1 p^{(1)},\dots,q_n p^{(n)}) = (q_1 p_1^{(1)},\dots,q_1p_{N_1}^{(1)},\dots, q_n p_{N_n}^{(n)} \]
    Size: $N_1+\dots+N_n$.
    \[ H(q_1 p^{(1)},\dots,q_np^{(n)}) = H(q) + \sum_{j=1}^n a_j H(p^{(j)} \]
  \end{claim}
  
  \begin{proof}
    If $n = 1$ and $q = 1$ there is nothing to prove.
    \begin{align*}
      H(q_1 p^{(1)},\dots,q_{n-1} p^{(n-1)}, q_n p^{(n)}) \\
      \overset{\text{Claim }1}{=} H(q_1 p^{(1)},\dots,q_{n-1}p^{(n-1)}, q_n) + q_n H(p^{(n)}) \\
      = H(q_n, q_1 p^{(1)},\dots,q_{n-2} p^{(n-2)}, q_{n-1}) 
      + q_{n-1} H(p^{(n-1)}) + q_n H(p^{(n)}) = \dots = \checkmark
    \end{align*}
  \end{proof}
\end{proof}



\end{document}
