\documentclass[a4paper]{article}

\usepackage[utf8]{inputenc}
\usepackage[english]{babel}
\usepackage{amsmath}
\usepackage{amsfonts}
\usepackage{amssymb}
\usepackage{hyperref}
\usepackage{amsthm}
\usepackage{stmaryrd}
\usepackage[nameinlink]{cleveref}
\usepackage{mathtools}
\usepackage{subfiles}
\usepackage{accents}
\usepackage{color}
\usepackage{enumitem}
\usepackage{csquotes}
\usepackage{titlesec}
\usepackage[colorinlistoftodos,prependcaption,textsize=footnotesize]{todonotes}

\hypersetup{
  pdfborder={0 0 0},
  colorlinks=true,
  linkcolor=darkgray,
  anchorcolor=darkgray
}

\newcommand{\lecture}{\vspace{5mm}\textcolor{blue}}
\newcommand{\interior}[1]{\accentset{\smash{\raisebox{-0.12ex}{$\scriptstyle\circ$}}}{#1}\rule{0pt}{2.3ex}}
\newcommand{\sectionbreak}{\clearpage}

\newcounter{cnt} \numberwithin{cnt}{section}

\newtheorem{defi}[cnt]{Definition}
\newtheorem*{defi*}{Definition}
\newtheorem*{rem}{Remark}
\newtheorem*{rem*}{Remark}
\newtheorem*{rems*}{Remarks}
\newtheorem{lemma}[cnt]{Lemma}
\newtheorem*{lemma*}{Lemma}
\newtheorem*{ex}{Example}
\newtheorem*{exs}{Examples}
\newtheorem*{ob}{Observation}
\newtheorem*{obs}{Observations}
\newtheorem{theorem}[cnt]{Theorem}
\newtheorem*{theorem*}{Theorem}
\newtheorem{prop}[cnt]{Proposition}
\newtheorem*{prop*}{Proposition}
\newtheorem{cor}[cnt]{Corollary}
\newtheorem*{cor*}{Corollary}
\newtheorem{nota}{Notation}
\newtheorem*{nota*}{Notation}
\newtheorem{claim}{Claim}

\crefname{rem}{Remark}{remarks}

\DeclareMathOperator{\prob}{\mathbb{P}}
\DeclareMathOperator{\expect}{\mathbb{E}}
\DeclareMathOperator{\vari}{\mathbb{V}}
\DeclareMathOperator{\covar}{Cov}
\DeclareMathOperator{\KLdiv}{D}
\DeclareMathOperator{\capa}{cap}
\DeclareMathOperator{\ape}{p_{\text{err}}^{(n)}}
\DeclareMathOperator{\supp}{supp}

\newcommand{\N}{\mathbb{N}}
\newcommand{\Z}{\mathbb{Z}}
\newcommand{\PP}{\mathbb{P}}
\newcommand{\C}{\mathbb{C}}
\newcommand{\Q}{\mathbb{Q}}
\newcommand{\R}{\mathbb{R}}

\newcommand{\overbar}[1]{\mkern 1.5mu\overline{\mkern-1.5mu#1\mkern-1.5mu}\mkern 1.5mu}
\newcommand{\inv}[1]{{#1}^{-1}}
\newcommand{\card}[1]{\left\vert{#1}\right\vert}
\newcommand{\set}[1]{\left\{#1\right\}}

\newcommand{\ubar}[1]{\underaccent{\bar}{#1}}

% use \abs{} for absolute values
\DeclarePairedDelimiter\abs{\lvert}{\rvert}%
\DeclarePairedDelimiter\norm{\lVert}{\rVert}%
\makeatletter
\let\oldabs\abs
\def\abs{\@ifstar{\oldabs}{\oldabs*}}
\let\oldnorm\norm
\def\norm{\@ifstar{\oldnorm}{\oldnorm*}}
\makeatother


\author{Martina Tscheckl}
\title{Mathematical foundation of information theory}

%++++++++++++++++++++++++++++++++++++++++++++++++++++++++++++++++++++++++++++++++++++++++++++++++++

\begin{document}
\maketitle
Please send feedback to \url{martina@tscheckl.eu}.
\tableofcontents

%---Chapter 1------------------------------------

\section{Introduction: Probability Theory}
\lecture{01.03.2016}

\subsection{Definitions and Axioms}

\begin{defi*}[Probability space]
  A \emph{probability space} is a triple $(\Omega,\mathcal{A},\PP)$ where 
  \begin{enumerate}
    \item $\Omega \neq \emptyset$ is a \emph{set},
    \item $\mathcal{A}$ is a \emph{$\sigma$-algebra} of subsets of $\Omega$,
    \item $\PP$ is a \emph{probability measure} on $\mathcal{A}$.
  \end{enumerate}
\end{defi*}

\begin{rems*}
  \begin{enumerate}[label=(\alph*)]
    \item The set $\Omega$ is called the \enquote{sample space} and contains \enquote{all possible aspects/cases related to an experiment}.
    \item By definition, $\mathcal{A} \subset \mathcal{P}(\Omega)$ where $\mathcal{P}(\Omega)$ is the collection of all subsets of $\Omega$. The set $\mathcal{P}(\Omega)$ is called the power set of $\Omega$.
    \item The elements $A \in \mathcal{A}$ are called \enquote{events}. 
    Events are observable (distinguishable) outcomes of an experiment.
  \end{enumerate}
\end{rems*}

\begin{exs}
  \begin{enumerate}[label=(\alph*)]
    \item one coin toss: $\Omega = \{ 0,1 \}$
    \item $500$ coin tosses: $\Omega = \{0,1 \}^{500}$
    \item one coin toss + final position of coin (+ number of spins):
    \[ \Omega = \{ 0,1 \} \times \R^2 ( \times \N_0 ) \]
  \end{enumerate}
\end{exs}

\begin{defi*}[Axioms]
  Let $(\Omega,\mathcal{A},\PP)$ be a probability space with $\PP:\mathcal{A}~\to~[0,1]$. 
  Then we define the following axioms:
  \begin{enumerate}
    \item $\emptyset \in \mathcal{A}$
    \item $A \in \mathcal{A} \implies A^C \in \mathcal{A}$
    \item If $(A_n)_{n \in \N}$ is a sequence in $\mathcal{A}$, then 
    \[ \bigcup_{i=1}^\infty A_n \in \mathcal{A} \text{.} \]
    \item $\PP(\Omega) = 1$
    \item $\sigma$-additivity: If $(A_n)$ is a sequence in $\mathcal{A}$ such that
    $A_k \cap A_l = \emptyset \: \forall k \neq l$ then
    \[ \PP(\biguplus_{n=1}^\infty A_n) = \sum_{n=1}^\infty \PP(A_n) \]
  \end{enumerate}
\end{defi*}

\begin{exs}
  Let's consider the examples (b) and (c) from above.
  \begin{enumerate}[label=(\alph*)]
    \item[(b)] $500$ coin tosses:
    \begin{align*}
      A &= [ \text{even number of $1$s} ] \\
      &= \{ \omega = (\omega_1,\dots,\omega_{500}) \in \Omega: \omega_1 + \dots + \omega_{500} \text{ is even} \} \text{.}
    \end{align*}
    \item[(c)] one coin toss + final position of coin + number of spins:
    \begin{align*}
      A &= [ \text{coin toss: } 1 \text{, positon from origin: distance } < 2m \text{, spins: } 13 ] \\
      &= \{ 1 \} \times \mathcal{B}((0,0),2) \times \{13\}
    \end{align*}
    where $\mathcal{B}((0,0),2) = \{(x,y) \in \R^2: \sqrt{x^2 + y^2} < 2 \}$.
  \end{enumerate}
\end{exs}

\begin{rem}  \todo{fix this remark}
  \begin{enumerate}
    \item If $\Omega$ is at most countable: usually $\mathcal{A} = \mathcal{P}(\Omega)$
    \item If $\Omega$ is uncountable, it is not good.
  \end{enumerate}
\end{rem}

\begin{ex}
  Let $\Omega = \R$. Then $\mathcal{A} = \mathcal{B}_{\R}$, where $\mathcal{B}_{\R}$ is the collection of all Borel sets on $\R$.
  So $\mathcal{A}$ is the smallest $\sigma$-algebra over $\R$ which contains all intervals.
\end{ex}

\begin{prop*}[Consequences of the axioms]
  Let $(\Omega,\mathcal{A},\PP)$ be a probability space with $\PP:\mathcal{A}~\to~[0,1]$.
  Then we can deduce from the axioms the following statements:
  \begin{enumerate}[label=(\alph*)]
    \item If $A_1,\dots,A_N \in \mathcal{A}$, then $\bigcup_{n=1}^N A_n \in \mathcal{A}$ and since $\PP(\emptyset) = 0$,
    \[ \PP(\biguplus_{n=1}^N A_n) = \sum_{n=1}^N \PP(A_n) \text{.} \]
    \item If $(A_n)_{n \in \N}$ is a finite or countable sequence in $\mathcal{A}$, then
    \[ \bigcap_{n=1}^\infty A_n \in \mathcal{A} \text{.} \]
    \item Let $A,B \in \mathcal{A}$. Then
    \[ \PP(A \cup B) = \PP(A) + \PP(B) - \PP(A \cap B) \text{.} \]
    \item Let $A,B \in \mathcal{A}$. If $A \subset B$, then 
    \[ \PP(A) \leq \PP(B) \text{.} \]
    \item \enquote{Continuity}: If $(A_n)_{n \in \N}$ is a sequence in $\mathcal{A}$ such that $\forall n$ $A_n \subset A_{n+1}$, then
    \[ \PP( \bigcup_{n=1}^\infty A_n ) = \lim_{n \to \infty} \PP(A_n) \text{.} \]
  \end{enumerate}
\end{prop*}

\begin{proof} \todo{fix/complete proof}
  \begin{enumerate}
    \item[(c)] Use $A \cup B = A \uplus (B \cap A^C)$.
    \todo{add pic}
    \item[(e)] \todo{add pic}
    \begin{align*}
      A_N &= A_1 \cup (A_2 \setminus A_1) \cup \dots \cup (A_N \setminus A_{N-1}) \\
      &= A_1 \cup \bigcup_{n=2}^N (A_n \setminus A_{n-1}) \\
      A &= A_1 \uplus \biguplus_{n=2}^\infty \underbrace{(A_n \setminus A_{n-1})}_{\in \mathcal{A}}
    \end{align*}
    \begin{align*}
      \PP(A) &= \PP(A_1) + \sum_{n=2}^\infty \PP (An \setminus A_{n-1}) \\
      &= \lim_{N \to \infty} (\PP(A_1) + \sum_{n=2}^N \PP(A_n \setminus \PP(A_{n-1})) \\
      &= \lim_{N \to \infty} \PP(A_N)
    \end{align*}
  \end{enumerate}
\end{proof}

\lecture{02.03.2016}

\end{document}
