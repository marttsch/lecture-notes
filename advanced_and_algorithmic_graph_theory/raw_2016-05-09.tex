\documentclass[aagt.tex]{subfiles}
\begin{document}
\lecture{09.05.2016}

The last theorem is not a characterization but a sufficient condition.

\lecture{09.05.2016}
Assume the geometrical dual $G^\ast$ of some graph $G$ has multiple edges. Thus, there exist at least 2 regions in $G$ with more than 1 common edge in their borders.
\todo{add pic}

Any pair of parallel edges in $G^\ast$ encloses a subgraph of $G$ which is connected to the rest exactly through the edges $e,f$ in $G$ which are the counter parts of $e^\ast,f^\ast$.
Thus, the remaining $e,f$ from $G$ disconnects that subgraph from the set of $F$.
Thus, $G$ is not 3-edge-connected. Thus, $G$ is not 3-connected.
Moreover, if $G$ is 3-connected (even if $G$ is 2-connected) \todo{add pic}, then $G^\ast$ has no loops either.
Thus, $G^\ast$ is simple.

\begin{ex}
  platonic solids
  \todo{add pic}
\end{ex}

\begin{defi*}
  A \emph{graph} is called \emph{polyhedral} if it is isomorphic to a 3-dimensional, convex, bounded polyhedron (bounded polyhedron = polytope) where the vertices, edges, facets of th polytope are the vertices, edges and region of the graph, respectively.
\end{defi*}

Homework: Polyhedral graphs are 3-connected (2-connectivity is trivial since every face is bounded by a cycle).
Polyhedral graphs are planar (construct a sphere in the interior of the polytope project the polytope on the surface of the sphere, and then from the sphere to the plane)

\begin{theorem}[4.12 Steinitz 1922]
  A graph is polyhedral if and only if it is planar and 3-connected.
\end{theorem}

\begin{proof}
  No proof.
\end{proof}

\begin{prop}
  If a graph is 3-connected, then its geometric dual is 3-connected.
\end{prop}

\begin{proof}[Sketch of the proof.]
  Let $S,T$ be two vertices in $G^\ast$ (i.e. two regions in $G$), $S \neq T$.
  Show that there exist 3 independent $S$-$T$-paths in $G^\ast$. Take two points such that in the interior of $S,T$ in $G$, respectively.
  \todo{add pic}
  Since $G$ is 3-connected: for all $v \in V(G)$ and for all 3 vertices on the border of $S$ ($T$), there exist 3 independent paths joining $v$ to the vertices on the border (fan theorem).
  connect $s$ ($t$) to 3 (arbitrarily chosen and then fixed) vertices on the border of $S$ ($T$).
  Let $G^\prime$ be the resulting graph.
  $G^\prime$ is 3 connected (e.g. 3 independent 1-v-paths (s-v-paths?) would be 3 independent  paths from v to the 3 vertices in the border (fan thm) extended by the three edges connecting $s$ to the border),
  Thus, there exist 3 independent s-t-paths in $G$.
  \todo{add pic}
  Consider a pair of these 3 paths and the area enclosed by them.
  There is a chain of regions which starts with $S$, ends with $T$, intersect this area such that any two consecutive regions in the chain are neighbouring. Such a chain in $G$ is an S-T-path in $G^\ast$.
  \todo{add pic}
  Any pair of path defines such a chain of regions, and hence an s-t-paths.
  The 3 chains defined by the 3 pairs of paths have no inner region in common.
  Therefore, the 3 corresponding s-t-paths are independent. (should be rigorously shown!)
\end{proof}

\begin{defi*}
  A polyhedral is called \emph{regular} (or \emph{platonic solid}) if aond only if it has conruent, regular, polygonal faces with the same number of faces meeting at each vertex.
\end{defi*}

Thus, a polyhedral graph corresponding to a platonic solid is regular.

\begin{theorem}[4.14]
  There are exactly 5 platonic solids.
\end{theorem}

\begin{proof}
  Consider the polyhedral graph corresopnding to a platonic solid.
  It is  regular; denote by $k$ the common degree of its vertices and by $h$ the number of vertices in each region. Apply Euler 
  \[ \abs{V} - \abs{E} + \abs{\mathcal{R}} = 2 \]
  together with $\abs{V} k = 2 \abs{E}$, $\abs{\mathcal{R}} k = 2 \mathcal{E}$.
  Then we get
  \[ \frac{2 \abs{E}}{k} - \abs{E} + \frac{2 \abs{E}}{h} = 2 \]
  Thus,
  \[ \frac{1}{k} + \frac{1}{h} - \frac{2 + \abs{E}}{2 \abs{E}} > \frac{1}{2} \]
  Furthermore, $k = \delta(G) \geq 3$ since $G$ is 3-connected, and $k \geq 3$ (number of edges on the border of a face (cycle)).
  Thus,
  \[ \frac{1}{k} > \frac{1}{2} - \frac{1}{h} \geq \frac{1}{2} - \frac{1}{3} = \frac{1}{6} \implies k \leq 5 \]
  and
  \[ \frac{1}{h} > \frac{1}{2} - \frac{1}{k} \geq \frac{1}{2} - \frac{1}{3} = \frac{1}{6} \implies h \leq 5 \]
  So we have $3 \leq h,k \leq 5$.
  Check all 9 pairs (h,k) and observe that only 5 of them correspond to the corresponding quantities in graphs.
\end{proof}

\begin{cor}
  The graphs of the platonic solids are exactly those polyhedral graphs which are regular toghether with their corresponding geometric duals.
\end{cor}

\begin{proof}
  Exercises.
\end{proof}

\begin{theorem}[4.15 Whitney 1934]
  Let $G$ be a planar embedding of $G$ and $G^\ast$ its geometric dual. 
  The set $C \subseteq E(G)$ is a cycle in $G$ if and only if the corresponding set of edges $C^\ast \subseteq E(G^\ast)$ is an inclusion-minimal separating set of edges.
\end{theorem}

\begin{proof}
  \begin{enumerate}
    \item[$\Rightarrow$] Let $C$ be a cycle in $G$. Consider the two areas of the plane in which $G$ is drawn, the area inside $C$ and the area outside $C$.
    $G^\ast$ has at least one vertex inside $C$ and at least one vertex outside $C$.
    \todo{add pic}
    Every path connecting square vertex inside $C$ to some vertex outside $C$ in $G^\ast$ has to intersect $C$ and contains hence an edge $e^\ast$ which the counter part of some edge $e \in E(C)$.
    Thus, $C^\ast$ is a separating set of edges in $G^\ast$.
    Is it inclusion-minimal? Yes, because the subgraph of $G^\ast$ inside $C$ and the subgraph of $G^\ast$ outside are both connected.
    \item[$\Leftarrow$] Let $C^\ast$ be an inclusion-minimal separating set of edges in $G^\ast$.
    Show that $C$ is a cycle.
    $G^\ast - C^\ast$ is disconnected. Thus, it has at least 2 connected components, \todo{add pic} and it cannot have more than 2, because otherwise the separating set was not inclusion minimal.
    Contract each connected component in one vertex and consider the dual of the contraction.
    \todo{add pic}
    The edges corresponding to $C^\ast$ underline{in this dual} build a cycle. \enquote{Remove} the contraction.
    The cycle $C$ is also a subgraph of the dual of the non-contracted $G^\ast$, so it is a subgraph of $G^\prime$.
  \end{enumerate}
\end{proof}

\begin{defi*}
  Let $G = (V,E)$ be a graph. $G^\ast = (V^\ast,E^\ast)$ is called \emph{a combinatorial dual} to $G$ if and only if there exists a bijection $\varphi: E \to E^\ast$ such that $C \subseteq E(G)$ is a cycle in $G$ if and only if $\varphi(C)$ is an inclusion minimal separating set of edges in $G^\ast$.
\end{defi*}

Does every graph in $G$ have a combinatorial dual?
Planar graphs do! (Whitney) \underline{Only} planar graphs do.

\begin{theorem}[Whitney]
  The only graphs which have a combinatorial dual are the planar graphs.
\end{theorem}

\begin{proof}[Sketch of the proof idea]
  4 Lemmas: Then the proof follows directly from Kuratowski
  \begin{lemma}\label{th_4_16_l_1}
    If $G$ has a combinatorial dual, then every subgraph of $G$ has a combinatorial dual.
  \end{lemma}
  \begin{lemma}\label{th_4_16_l_2}
    Let $G$ be a graph and $G^\ast$ a combinatorial dual of $G$. Let $e_0,e_1 \in E(G)$.
    Let $e_0^\ast,e_1^\ast$ be the edges corresponding to $e_0,e_1$ in $G^\ast$, respectively.
    $e_0^\ast,e_1^\ast$ are parallel if and only if every cycle containing one of $e_0,e_1$ contains also the other.
  \end{lemma}
  \begin{lemma}\label{th_4_16_l_3}
    Let $H$ be a subdivision of $G$. If $H$ has a combinatorial dual, then also $G$ has one.
  \end{lemma}
  \begin{lemma}\label{th_4_16_l_4}
    Neither $K_{3,3}$ nor $K_5$ have a combinatorial dual.
  \end{lemma}
\end{proof}


\underline{Recognition Problem}\\
Instance: A graph $G=(V,E)$\\
Question: Is $G$ planar?

Kuratowski gives a characterisation which would lead to an exponential algorithm.
Solvale in linear time if in $\bigO(n+m)$.

J.Hopcraft and R. Tarjan, Efficient planarity testing, Journal of ACM (Association of computer machinery) 21 (4), 549-568, 1974.

\underline{State of the are algorithm}\\
which goes back to J.M. Voyer and W.J. Myrvold, On the cutting edge simplified $bigO(n)$ planarity by edge addition,
Journal of Graph Algorithm and Applications 8(3), 241-273, 2004. (volume 8, issue 3)


\end{document}
