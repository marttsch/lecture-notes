\documentclass[aagt.tex]{subfiles}
\begin{document}
\lecture{21.04.2016}

\begin{proof}
$(\Rightarrow)$ $0 < a_1 \leq a_2 \leq \dots \leq a_n < n$, $a_i \in \N$ for all $1 \leq i \leq n$ and
$a_i \leq i \iff a_{n-i} \geq n-i$ for all $i < \frac{n}{2}$
Thus, $(a_i)_{1\leq i \leq n}$ ham seq
By condtradiction there exists a $G$ with degree sequence $(d_i)_{1\leq i \leq n}$, $d_i \geq a_i$ for all $1 \leq i \leq n$ and $G$ non ham

We showed $d_i \leq i \implies d_{n-i} \geq n-i$ for all $i < \frac{n}{2}$ (\ref{0418_2})

Let $x,y \in V(G)$ such that $\{x,y\} \notin E(G)$ with max $\de(x) + \de(y)$, wlog $\de(x) \leq \de(y)$.
Thus, $G \cup \{x,y\}$ is ham and every cycle $H$ in $G \cup \{x,y\}$ contains $\{x,y\}$.
\todo{add pic}
$H - \{x,y\}$ is a ham path in $G$. Denote it by $x_1 \coloneqq x, x_2,\dots,x_n \coloneqq y)$
\todo{add pic}
As in proof of Dirac's theorem 
\begin{align*}
  I &\coloneqq \{i: 1 \leq i \leq n-1, \{x_1,x_{i-1}\} \in E\} \\
  J &\coloneqq \{i: 1 \leq i \leq n-1, \{x_i,x_n\} \in E \}
\end{align*}

$I \cup J \subseteq \{1,2,\dots,n-1\}$, $I \cap J = \emptyset$ (as in Dirac's theorem)
\begin{align}\label{0421_4}
  \de_G(x) + \de_G(y) = \abs{I} + \abs{J} = \abs{I \cup J} \leq n-1
\end{align}

Recall
\begin{align}\label{0421_3}
  \de_G(x) = \de(x) < \frac{n}{2}
\end{align}

For all $i \in I$
\[ \de(x_i) \leq \de(x) \]
because otherwise $\de(x_i) + \de(y) > \de(x) + \de(y)$ and $\{x_i,y\} \notin E(G)$, so instead of $x$ and $y$ we would have chosen $x_i$ and $y$ $\lightning$.

So $G$ contains at least $\abs{I}$ vertices with degree $\leq \de(x) \eqqcolon h = \abs{I}$.
Thus, $d_h \leq h \overset{\ref{0418_2}}{\implies} d_{n-h} \geq n-h$
\todo{add pic}

There exists at least one vertex among these $h+1$ ($d_{n-h} \leq d_{n-h+1} \leq \dots \leq d_n$) which is non-adjacent to $x$.
Let this vertex be $z$, so $\{x,z\} \notin E(G)$.
\[ \de(x) + \de(z) \geq h + n-h = n \]
which contradicts \ref{0421_4}.

So $(a_i)_{1\leq i\leq n}$ has to be a hamilt seq

$(\Leftarrow)$ $(a_i)_{1 \leq i \leq n}$ is hamilt $\implies$ condition \ref{0418_star} is fulfilled
or equivalently, condition \ref{0418_star} is not fulfilled $\implies$ $(a_i)_{1 \leq i \leq n}$ is not hamilt, i.e. there exists a graph $G$ with degree sequence $d_i \geq a_i$ for all $i$ which is not hamilt.

Construct such a graph for the sequence $a_i$ which violated the condition \ref{0418_star} at index $h$, $a_h \leq h$ and $a_{n-h} \leq n-h-1$.
Consider 
\begin{align}\label{0421_5}
  (\underbrace{h,h,\dots,h}_{h \text{ times}},\underbrace{n-h-1,\dots,n-h-1}_{n-2h \text{ times}},\underbrace{n-1,\dots,n-1}_{h \text{ times}})
\end{align}
\begin{align*}
  k &\leq n-h-1 \\
  2h &\leq n-1 (\text{ because } h < \frac{n}{2}
\end{align*}

So the sequence is non decreasing and of length $n$.

Construct $G$ having the above mentioned sequence as the degree sequence
\todo{add pic}

\begin{align*}
  E = \{\{v_i,v_j\}: i,j > h\} \cup \{\{v_i,v_j\}: 1 \leq i \leq h, j > n-h \}
\end{align*}

\enquote{$G$ is complete over red and green vertices. $G$ is complete bipartite over blue on on one side and green on the other siede.}

Show degree seq of is
First,
\begin{align*}
  \de(v_i) &= h \;\; \forall 1 \leq i \leq k \text{ (blue vertices)} \\
  \de(v_i) &= n-h-1 \;\; \forall h+1 \leq i \leq n-h \text{ (red vertices)} \\
  \de(v_i) &= h+n-h-1 = n-1 \;\; \forall n-h+1 \leq i \leq n \text{ (green vertices)}
\end{align*}

Second, pic
\todo{add pic}
Red vertices are missing because the successor of every blue vertex is a green vertex, and the are exactly as many green ($h$) vertices as blue vertices, and also the predecessor of every blue vertex is a green vertex

\begin{align*}
  d_{n-h} &\geq n-h \text{ and} \\
  d_h > h
\end{align*}

\begin{rem}[Exercise]
  Insert exercise here. Analogous construction like above
\end{rem}
\end{proof}

\section{4. Topological graph theory; planar graph}

History of this theory:
\begin{enumerate}
  \item Euler formula
  \item Theorem of Kuratowski
  \item four colours conjecture and then theorem
  \item algorithmic aspects: a number of NP-hard opt. problems in general graphs are efficiently solvable or approximable in planar graphs (max cut, k-multi cut, colouring problem, stable set,...)
\end{enumerate}

\subsection{4.1 Definitions and elementary concepts}

\begin{defi*}
  A graph $G$ is called \emph{planar} if there exists an embedding (or representation) of $G$ in the plane such that
  \begin{enumerate}
    \item the vertices of $G$ are points in the plane
    \item the edges of $G$ are Jordan curves, such that 
    \begin{enumerate}
      \item every Jordan curve intersects the set of points representing vertices only on its endpoint, and
      \item any two different Jordan curves can intersect only at some endpoints.
    \end{enumerate}
  \end{enumerate}
  A \emph{plane graph} is just a representation of a planar graph as described above in the plane.
  \todo{add pic}
\end{defi*}

\begin{defi*}
  A \emph{Jordan curve} is (the image of) a homeomorphic mapping $f$ of $I_1 = [0,1]$ into a topological space,
  i.e. $f: [0,1] \to A$ (top. space) such that $f$ is bijective, continuous and $\inv{f}$ is also continuous.
\end{defi*}

We will prove all results in this chapter for the topological space $(\R^2,O)$ where the basis $O$ of the open sets is
\[ O \coloneqq \{B(x,r): x \in \R^2, r>0\} \]
with $B(x,r) \coloneqq \{y \in R^2: l_2(x,y) < r \}$.

\begin{rem}
  Dimension is 1. $S_1$ is cycle of radius $1$, $P$ the North Pole
  \todo{add pic}
  $f: S_1 \setminus \{P\} \to \R$, $x \mapsto f(x)$
  Stereographic projection $S_1 \setminus \{P\} \ni \R$, $S_2 \setminus \{P\} \ni \R^2$
  \todo{add pic}
\end{rem}

\begin{defi*}
  If the edges are removed from a planar embedding of a planar graph, then the plane would decompose in connected areas which are called \emph{regions} (or \emph{face}) of which exactly one is unbounded.
  This unbounded region is called \emph{outer region}. The \emph{boarder of a region} consists of  all edges which are contained in the (topol.) closure of the region.
\end{defi*}

\begin{rem}
  Consider an arbitrary plane embedding of a planar graph $G$ and a face $F$ which is not the outer face.
  There exists another plane embedding $H_2$ of $G$ in which the image of $f$ is the outer face.
  \todo{add pic}
  By stereographic projection place a sphere \enquote{on the plane} such that it touces the plane where $H_1$ lies
  and do the projection $H_1$ on this sphere. Rotate sphere s. t. the nothe pole becomes the touching point that lies on $F$, project the sphere embedding back to the plane. This is the required $H_2$.
\end{rem}

\begin{prop}[4.1]\label{p_4_1}
  A graph is planar if and only if all its blocks are planar.
\end{prop}

\begin{proof}
  wlog assume that $G$ is connected, otherwise ... with connected components am embedded there s.t. the embeddings are \enquote{far apart}.
  
  Consider $\bc(G)$, we know it is a face.
  \begin{enumerate}
    \item[$\Rightarrow$] trivial
    \item[$\Leftarrow$] Assume every block is planar, Show that $G$ is planar.
    Pick up an arbitrary block, say $b_1$, and construct one plane embedding of it.
    Do a DFS (or BFS...) on $\bc(G)$ starting at $b_1$.
    Visit the other blocks of $G$ in the order implied by this search.
    and every time a new block is encountered construct a planar embedding oth this block such that the cur vertex from which th block is entered has on the boarder of the outer face of the embedding construction so far.
    \todo{add pic}
    (Maybe the embedding constructed so far needs to be transformed so that the cut vertex mentioned above lies on its outer face.)
  \end{enumerate}
\end{proof}

\begin{prop}\label{p_4_2}
  A planar graph is 2-connected if and only if the boarder of each face of any plane embedding of $G$ is a cycle.
\end{prop}

\begin{proof}
  \begin{enumerate}
    \item[$\Leftarrow$] \todo{add pic}
    \item[$\Rightarrow$] Exercise
  \end{enumerate}
\end{proof}

\end{document}
